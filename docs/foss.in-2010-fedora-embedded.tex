% Created 2010-12-17 Fri 01:06
\documentclass[11pt]{beamer}

      \mode<presentation>

      \usetheme{Copenhagen}

      \usecolortheme{lily}

      \beamertemplateballitem

      \setbeameroption{show notes}
      \usepackage[utf8]{inputenc}

      \usepackage[T1]{fontenc}

      \usepackage{hyperref}

      \usepackage{color}
      \usepackage{listings}
      \lstset{numbers=none,language=[ISO]C++,tabsize=4,
  frame=single,
  basicstyle=\small,
  showspaces=false,showstringspaces=false,
  showtabs=false,
  keywordstyle=\color{blue}\bfseries,
  commentstyle=\color{red},
  }

      \usepackage{verbatim}

      \institute{Fedora}
          
       \subject{RMRF}



\title{foss.in/2010 - Fedora Embedded and QEMU/ARM hands on}
\author{Saleem Ansari}
\date{17th December 2010}

\begin{document}

\maketitle

\setcounter{tocdepth}{3}
\tableofcontents
\vspace*{1cm}


\section{Agenda}
\label{sec-1}


\begin{frame}[fragile]\frametitle{Agenda}
\label{sec-1.1}

\begin{itemize}

\item Fedora Embedded SIG – Special Interest Group\\
\label{sec-1.1.1}


\item Some background knowlege\\
\label{sec-1.1.2}


\item A basic hands-on with QEMU/ARM\\
\label{sec-1.1.3}


\item Fedora ARM Infrastructure\\
\label{sec-1.1.4}


\end{itemize} % ends low level
\end{frame}
\section{Fedora Embedded SIG – Special Interest Group}
\label{sec-2}


\begin{frame}[fragile]\frametitle{Fedora Embedded SIG – Special Interest Group}
\label{sec-2.1}

\begin{itemize}

\item Goals - \href{http://fedoraproject.org/wiki/SIGs/Embedded}{http://fedoraproject.org/wiki/SIGs/Embedded}\\
\label{sec-2.1.1}

\begin{itemize}

\item Create high quality packages of cross-compilers and related tools\\
\label{sec-2.1.1.1}


\item Provide packaging guidelines\\
\label{sec-2.1.1.2}


\item Make distribution releases on sub architectures\\
\label{sec-2.1.1.3}

\end{itemize} % ends low level

\item Targeted platforms/arches: ARM, AVR32, AVR, Blackfin, TIGCC, RTEMS, others\\
\label{sec-2.1.2}


\item For this talk we will focus on ARM architecture\\
\label{sec-2.1.3}


\end{itemize} % ends low level
\end{frame}
\begin{frame}[fragile]\frametitle{Why Fedora on ARM architecture? Multiple reasons!}
\label{sec-2.2}

\begin{itemize}

\item 2007: \~{} 98\% of one billion mobile phones sold each year use at least one ARM processor.\\
\label{sec-2.2.1}


\item 2009: \~{} 90\% of all embedded 32-bit RISC processors are ARM processors.\\
\label{sec-2.2.2}


\item 2010: \~{} 5 Billion ARM9 processors have been shipped so far\\
\label{sec-2.2.3}


\item ARM processors to enter server markets - Marvell's announcement\\
\label{sec-2.2.4}


\item ARM processor- Eagle\\
\label{sec-2.2.5}

\begin{itemize}

\item based on Cortex A15 architecture, it has hardware virtualization support!\\
\label{sec-2.2.5.1}

\end{itemize} % ends low level

\item OLPC is a one big project which interested to use Fedora ARM port\\
\label{sec-2.2.6}


\end{itemize} % ends low level
\end{frame}
\section{Some background}
\label{sec-3}


\begin{frame}[fragile]\frametitle{Some background}
\label{sec-3.1}

\begin{itemize}

\item Platform / processor selection for a Linux/embedded project\\
\label{sec-3.1.1}

\begin{itemize}

\item at-least 32bit processor\\
\label{sec-3.1.1.1}


\item processor should have a MMU\\
\label{sec-3.1.1.2}


\item non-MMU based processors can use ucLinux\\
\label{sec-3.1.1.3}

\begin{itemize}

\item however you have to modify applications too\\
\label{sec-3.1.1.3.1}

\end{itemize} % ends low level

\item floating point unit\\
\label{sec-3.1.1.4}

\begin{itemize}

\item which optional and can be emulated by libraries\\
\label{sec-3.1.1.4.1}


\end{itemize} % ends low level
\end{itemize} % ends low level
\end{itemize} % ends low level
\end{frame}
\begin{frame}[fragile]\frametitle{Why do we need a boot loader?}
\label{sec-3.2}

\begin{itemize}

\item There is a lot of work done by boot loader.\\
\label{sec-3.2.1}


\item Kernel assumes that a lot of system initialization is already done:\\
\label{sec-3.2.2}

\begin{itemize}

\item configuring memory sub-system.\\
\label{sec-3.2.2.1}


\item loading kernel image at the correct memory address\\
\label{sec-3.2.2.2}


\item loading initial RAM disk ( optional )\\
\label{sec-3.2.2.3}


\item initializing boot parameters\\
\label{sec-3.2.2.4}


\item obtaining the ARM Linux machine type\\
\label{sec-3.2.2.5}


\item initializing register variables for entry into kernel\\
\label{sec-3.2.2.6}

\end{itemize} % ends low level

\item Writing code for all this is:\\
\label{sec-3.2.3}

\begin{itemize}

\item highly error prone, repetitive, cumbersome and boring\\
\label{sec-3.2.3.1}


\item a readymade bootloader generally just fits in\\
\label{sec-3.2.3.2}


\end{itemize} % ends low level
\end{itemize} % ends low level
\end{frame}
\begin{frame}[fragile]\frametitle{Why do we need Linux Kernel?}
\label{sec-3.3}

\begin{itemize}

\item Linux, of-course, is Open Source\\
\label{sec-3.3.1}


\item Vendor Independence\\
\label{sec-3.3.2}


\item POSIX compliance\\
\label{sec-3.3.3}


\item Varied Hardware Support\\
\label{sec-3.3.4}


\item linux kernel has many device drivers already built by someone\\
\label{sec-3.3.5}

\begin{itemize}

\item provide us with well defined device interfaces\\
\label{sec-3.3.5.1}


\item does the grunt work of hardware level interfacing with the devices\\
\label{sec-3.3.5.2}


\item driver works on another hardware with usually very few changes\\
\label{sec-3.3.5.3}


\end{itemize} % ends low level
\end{itemize} % ends low level
\end{frame}
\section{A basic hands-on with QEMU/ARM}
\label{sec-4}


\begin{frame}[fragile]\frametitle{About QEMU}
\label{sec-4.1}

\begin{itemize}

\item QEMU is an emulator for popular CPU\\
\label{sec-4.1.1}

\begin{itemize}

\item x86, ARM, PowerPC, Sparc32/64, MIPS and ColdFire\\
\label{sec-4.1.1.1}


\item AVR32 support is in progress at \href{http://repo.or.cz/w/qemu/avr32.git}{http://repo.or.cz/w/qemu/avr32.git}\\
\label{sec-4.1.1.2}


\item It also simulates peripheral devices, basically a complete System on Chip.\\
\label{sec-4.1.1.3}

\end{itemize} % ends low level

\item What makes QEMU special for doing embedded systems development?\\
\label{sec-4.1.2}

\begin{itemize}

\item No need of actual hardware for prototyping.\\
\label{sec-4.1.2.1}


\item Its greener- no need of any power source.\\
\label{sec-4.1.2.2}


\item Many projects already use QEMU\\
\label{sec-4.1.2.3}

\begin{itemize}
\item Android, MeeGo, etc.
\end{itemize}
\end{itemize} % ends low level
\end{itemize} % ends low level
\end{frame}
\begin{frame}[fragile]\frametitle{Hands-on. We will discuss the following:}
\label{sec-4.2}

\begin{itemize}

\item Development setup up\\
\label{sec-4.2.1}

\begin{itemize}

\item ARM toolchain, QEMU etc.\\
\label{sec-4.2.1.1}

\end{itemize} % ends low level

\item What happens at system reset?\\
\label{sec-4.2.2}

\begin{itemize}

\item Initialization.\\
\label{sec-4.2.2.1}


\item Bootloader.\\
\label{sec-4.2.2.2}


\item Kernel.\\
\label{sec-4.2.2.3}


\item Initial root filesystem.\\
\label{sec-4.2.2.4}


\end{itemize} % ends low level
\end{itemize} % ends low level
\end{frame}
\begin{frame}[fragile]\frametitle{System setup}
\label{sec-4.3}

\begin{itemize}

\item Fedora ARM Toolchain\\
\label{sec-4.3.1}

\begin{itemize}
\item ARMv5 only - \href{http://fedoraproject.org/wiki/Architectures/ARM/CrossToolchain}{http://fedoraproject.org/wiki/Architectures/ARM/CrossToolchain}
\end{itemize}

\item QEMU/ARM\\
\label{sec-4.3.2}


\item mkimage - tool to create boot images\\
\label{sec-4.3.3}


\item other tools\\
\label{sec-4.3.4}

\begin{itemize}
\item bridge-utils ( optional )
\item koji
\end{itemize}
\end{itemize} % ends low level
\end{frame}
\begin{frame}[fragile]\frametitle{What happens when an ARM based board is powered ON?}
\label{sec-4.4}

\begin{itemize}

\item Understand the memory map for RAM and ROM\\
\label{sec-4.4.1}


\item Startup code\\
\label{sec-4.4.2}

\begin{itemize}

\item First instruction\\
\label{sec-4.4.2.1}


\item Vector table\\
\label{sec-4.4.2.2}


\item Stack initialization\\
\label{sec-4.4.2.3}


\item Relocating the RAM to a different address\\
\label{sec-4.4.2.4}


\item Running from RAM\\
\label{sec-4.4.2.5}


\end{itemize} % ends low level
\end{itemize} % ends low level
\end{frame}
\begin{frame}[fragile]\frametitle{Hands on:}
\label{sec-4.5}

\begin{itemize}

\item Target hardware - VersatilePB - emulated QEMU\\
\label{sec-4.5.1}


\item Lets make a minimalistic linux based system on ARM\\
\label{sec-4.5.2}


\item Following steps are involved:\\
\label{sec-4.5.3}

\begin{itemize}
\item Compile and configure U-Boot boot-loader
\item Compile and configure Linux Kernel
\item Create a root file-system

\begin{itemize}
\item We will use Busybox for shell command interface
\end{itemize}

\item Shove all of them into a one big chunk to be flashed onto memory
\end{itemize}
\end{itemize} % ends low level
\end{frame}
\begin{frame}[fragile]\frametitle{What we learn from this excercise?}
\label{sec-4.6}

\begin{itemize}

\item Its a great learning experience\\
\label{sec-4.6.1}


\item However its not productive in the long term.\\
\label{sec-4.6.2}


\item It doesn't scale to thousands of software packages.\\
\label{sec-4.6.3}


\item Fedora ARM port solves this probelm:\\
\label{sec-4.6.4}

\begin{itemize}

\item Proven packaging technology already in use in many of the most popular distros\\
\label{sec-4.6.4.1}


\item Reduce development time and time to market for your next device\\
\label{sec-4.6.4.2}


\end{itemize} % ends low level
\end{itemize} % ends low level
\end{frame}
\section{Fedora ARM Infrastructure}
\label{sec-5}


\begin{frame}[fragile]\frametitle{Fedora ARM Infrastructure}
\label{sec-5.1}

\begin{itemize}

\item Seneca Centre for Development of Open Technology (CDOT) at the School of Computer Studies at Seneca College, Toronto.\\
\label{sec-5.1.1}


\item \href{http://zenit.senecac.on.ca/wiki/index.php/Fedora_ARM_Secondary_Architecture}{http://zenit.senecac.on.ca/wiki/index.php/Fedora\_ARM\_Secondary\_Architecture}\\
\label{sec-5.1.2}


\item \href{http://zenit.senecac.on.ca/wiki/index.php/Fedora_ARM_Koji_Buildsystem}{http://zenit.senecac.on.ca/wiki/index.php/Fedora\_ARM\_Koji\_Buildsystem}\\
\label{sec-5.1.3}


\item Latest news:\\
\label{sec-5.1.4}

\begin{itemize}

\item \href{http://fedora-arm.blogspot.com/}{http://fedora-arm.blogspot.com/}\\
\label{sec-5.1.4.1}


\item \href{http://paulfedora.wordpress.com/}{http://paulfedora.wordpress.com/}\\
\label{sec-5.1.4.2}

\end{itemize} % ends low level

\item Current status page\\
\label{sec-5.1.5}

\begin{itemize}

\item \href{http://arm.koji.fedoraproject.org/status/}{http://arm.koji.fedoraproject.org/status/}\\
\label{sec-5.1.5.1}

\end{itemize} % ends low level

\item Recent video by Paul Whalen\\
\label{sec-5.1.6}

\begin{itemize}

\item \href{http://fsoss.senecac.on.ca/2010/node/24}{http://fsoss.senecac.on.ca/2010/node/24}\\
\label{sec-5.1.6.1}


\end{itemize} % ends low level
\end{itemize} % ends low level
\end{frame}
\begin{frame}[fragile]\frametitle{Bulding packages for Fedora ARM}
\label{sec-5.2}

\begin{itemize}

\item Create or fix an existing SRPM package\\
\label{sec-5.2.1}


\item Schedule a build on ARM Koji to create RPM\\
\label{sec-5.2.2}


\item Check and fix if necessary\\
\label{sec-5.2.3}


\item ARM Koji web interface\\
\label{sec-5.2.4}


\item More on Howto use koji\\
\label{sec-5.2.5}

\begin{itemize}

\item \href{http://fedoraproject.org/wiki/Koji}{http://fedoraproject.org/wiki/Koji}\\
\label{sec-5.2.5.1}


\end{itemize} % ends low level
\end{itemize} % ends low level
\end{frame}
\begin{frame}[fragile]\frametitle{Contributing to Fedora/ARM}
\label{sec-5.3}

\begin{itemize}

\item fedoraproject.org/wiki/Architectures/ARM/HowToQemu\\
\label{sec-5.3.1}


\item Mailing list\\
\label{sec-5.3.2}

\begin{itemize}

\item \href{https://admin.fedoraproject.org/mailman/listinfo/arm}{https://admin.fedoraproject.org/mailman/listinfo/arm}\\
\label{sec-5.3.2.1}

\end{itemize} % ends low level

\item IRC - \#fedora-arm on irc.freenode.net\\
\label{sec-5.3.3}


\item Issues not yet resolved:\\
\label{sec-5.3.4}

\begin{itemize}

\item \href{https://bugzilla.redhat.com/show_bug.cgi?id=ARMTracker}{https://bugzilla.redhat.com/show\_bug.cgi?id=ARMTracker}\\
\label{sec-5.3.4.1}

\end{itemize} % ends low level

\item More info at:\\
\label{sec-5.3.5}

\begin{itemize}

\item \href{http://fedoraproject.org/wiki/Architectures/ARM/Team_and_Developers}{http://fedoraproject.org/wiki/Architectures/ARM/Team\_and\_Developers}\\
\label{sec-5.3.5.1}


\end{itemize} % ends low level
\end{itemize} % ends low level
\end{frame}
\begin{frame}[fragile]\frametitle{Current state of affairs in Fedora/ARM}
\label{sec-5.4}

\begin{itemize}

\item Fedora 13 release for ARM will hopefully be near Xmas.\\
\label{sec-5.4.1}


\item The default toolchain is still for ARMv5 and doesn't support hardfp so may of the modern ARM CPUs aren't optimally used\\
\label{sec-5.4.2}


\item Floating point support in the Fedora/ARM cross toolchain:\\
\label{sec-5.4.3}

\begin{itemize}

\item Discussion is on for which of soft/softfp or hardfp to select for upcoming ARMv7 toolchain\\
\label{sec-5.4.3.1}

\end{itemize} % ends low level

\item OLPC (currently the only one eager to use Fedora ARM for a large scale project)\\
\label{sec-5.4.4}

\begin{itemize}

\item they seem to be aiming to jump from F-11 to F-14, for their next release\\
\label{sec-5.4.4.1}


\item its not clear if that includes XO-1.75 ( the OLPC's ARM target ).\\
\label{sec-5.4.4.2}



\end{itemize} % ends low level
\end{itemize} % ends low level
\end{frame}
\section{Questions?}
\label{sec-6}


\begin{frame}[fragile]\frametitle{Questions?}
\label{sec-6.1}


\end{frame}
\begin{frame}[fragile]\frametitle{Thank you!}
\label{sec-6.2}

\begin{itemize}

\item tuxdna at \#fedora-arm on irc.freenode.net\\
\label{sec-6.2.1}

\end{itemize} % ends low level
\end{frame}

\end{document}
